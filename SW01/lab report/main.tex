\documentclass[onecolumn]{article}
\usepackage{url}
\usepackage{algorithmic}
\usepackage[a4paper]{geometry}
\usepackage{datetime}
\usepackage[margin=2em, font=small,labelfont=it]{caption}
\usepackage{graphicx}
\usepackage{mathpazo} % use palatino
\usepackage[scaled]{helvet} % helvetica
\usepackage{microtype}
\usepackage{amsmath}
\usepackage{subfigure}
\usepackage{enumitem} % alphabetic enumeration
% Letterspacing macros
\newcommand{\spacecaps}[1]{\textls[200]{\MakeUppercase{#1}}}
\newcommand{\spacesc}[1]{\textls[50]{\textsc{\MakeLowercase{#1}}}}

\title{\spacecaps{Lab report: SW01 }\\ \normalsize \spacesc{TSM\_DeLearn} }

\author{Andrin Bürli\thanks{andrin.buerli@hslu.ch}, Nursinem Dere\thanks{nursinem.dere@stud.hslu.ch}, Fabian Gröger\thanks{fabian.groeger@hslu.ch}\\Hochschule Luzern}
\date{\today}

\begin{document}
\maketitle

\section{Exercise 5: Review Questions}
\subsection{Supervised vs. unsupervised systems}

\begin{enumerate}[label=(\alph*)]
	\item Supervised learning: Weather or not an email is spam is a specific and given label. The text / metadata of the email can be considered as features.
	\item Unsupervised learning: We do not have any given labels, but rather want to find some structure / clusters in the dataset.
	\item Unsupervised learning: Again no given labels, we want to find structure in terms of segments / clusters.
	\item Supervised learning: We want to classify the patient to give him a diagnose, on basis on some given labled dataset.
\end{enumerate}

\subsection{Classification vs. regression systems}
Yes, it is possible to transform a regression problem into a classification problem. One of the straight forward ways is to group/bin the continuous variable to get a discrete one. The benefit is that this creates more data for a single target variable. However, the disadvantage is that the target variable isn't as accurate as before. 

\subsection{Perceptron}
\begin{enumerate}[label=(\alph*)]
\item  The perceptron can be used for both classification and linear regression.
\item In case of classification if the problem (dataset) is linearly seperatable it will converge.
\item Models with only a single (no hidden) layer of perceptrons.
\item In case of the XOR problem the Perceptron will not converge, because there exists no linear function which seperates the two target classes.

\end{enumerate}

\end{document}

